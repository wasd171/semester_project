\chapter{Validation}

Two parameters defining the assembled resonator are the central frequency $\omega_0$ and the quality factor $Q$. In this chapter we measure their experimental values and compare to those defined in the table \ref{tbl:restrictions_siverns}.

\section{Measurements}

In order to determine $\omega_0$ and $Q$ we would take a look at the reflection spectrum of the system consisting of resonator + coaxial cable + capacitor. For every configuration the depth of the resonance was optimized by changing the distance and the angle between the antenna mount \ref{subsection:antenna_mount} and the top cap \ref{subsection:cap_top}, effectively adjusting the antenna to coil coupling. 

\begin{table}[h]
\centering
\begin{tabular}{| l | l | r | r |}
	\hline
	Parameter & Definition & Values & Units\\
	\hline \hline
	$C_{load}$ & External capacitive load & $10,15,22$ & pF\\
	\hline
	$L_{coax}$ & Length of the coaxial cable & $10, 20, 50$ & cm\\
	\hline
\end{tabular}
\label{tbl:experimental_parameters}
\caption{Additional experimental parameters}
\end{table}

At the time of writing there is no clear values for the capacitance of the trap and the length of the wires connecting it with the resonator. By using a set of parameters defined in the table \ref{tbl:experimental_parameters} we aim to measure $\omega_0$ and $Q$ in the region similar to the one used for the numerical calculations and broad enough to include the point $\{C_{load} = C_{trap}; L_{coax}\}$ of the final setup.

\begin{itemize}
	\item For $L_{coax}=50$ two coaxial cables were connected, introducing 2 additional SMA connectors to the contour
	\item Available SMA connectors were of the male type. Unfortunately it is also the type commonly used for the wires, so we had to use an additional female-female adapter (+1 for $L_{coax} = 50$)
\end{itemize}

\begin{figure}[h]
	\centering
	\includegraphics[width=\textwidth]{images/R_plot_C_10}
	\label{fig:R_C_10}
	\caption{Reflection spectrum for $C = 10$}
\end{figure}
\begin{figure}[h]
	\centering
	\includegraphics[width=\textwidth]{images/R_plot_C_15}
	\label{fig:R_C_10}
	\caption{Reflection spectrum for $C = 15$}
\end{figure}
\FloatBarrier
\begin{figure}[h]
	\centering
	\includegraphics[width=\textwidth]{images/R_plot_C_22}
	\label{fig:R_C_10}
	\caption{Reflection spectrum for $C = 22$}
\end{figure}
\origsection{Analysis}
\FloatBarrier
Finding $\omega_0$ from experimental data is trivial, it is the central frequency of the peak. For the transmission spectrum $Q$ can be defined as
\begin{equation}
	Q = \frac{\omega_0}{\Delta \omega},
\end{equation}
where $\Delta \omega$ is the width of the peak at the $1/\sqrt{2}$ height. Since we are working with the reflection spectrum we need to adjust the height using the following expression
\begin{equation}
	R_{1/\sqrt{2}} = 1 - \frac{1 - R_{peak}}{\sqrt{2}}.
	\label{eq:reflection_height}
\end{equation}

Measured values of $\omega_0$ and $Q$ are presented in the table \ref{tbl:Q_w_results}. Visual comparison with the predicted values from the table \ref{tbl:restrictions_siverns} is shown in the chart \ref{fig:Q_w_deviation}. $C_{load} = 22$ pF is not equal to the simulated $C_{trap} = 20$ pF but was considered being close enough. The closer experimental data is to the point $\{0; 0\}$ the better it reflects the simulations. It can be seen that increased capacitance due to both $C_{load}$ and $L_{coax}$ tends to increase deviations from predictions.
\begin{table}[h]
\centering
\begin{tabular}{| r | r || r | r || r | r |}
	\hline
	$C_{load}$, pF & $L_{coax}$, cm & $\omega^{simul}_0$, $2\pi*$MHz & $\omega_0$, $2\pi*$MHz & $Q^{simul}$ & $Q$\\
	\hline \hline
	10 & 10 & 44.6 & 44.7 & 418 & 223.5\\
	\hline
	10 & 20 & \dittotikz & 36.3 & \dittotikz & 269.9\\
	\hline
	10 & 50 & \dittotikz & 22.3 & \dittotikz & 140.4\\
	\hline
	15 & 10 & 45.6 & 40.6 & 360 & 216.5\\
	\hline
	15 & 20 & \dittotikz & 33.9 & \dittotikz & 268.6\\
	\hline
	15 & 50 & \dittotikz & 21.6 & \dittotikz & 69.7\\
	\hline
	22 & 10 & 46.9 & 36.5 & 311 & 153.5\\
	\hline
	22 & 20 & \dittotikz & 31.2 & \dittotikz & 192.3\\
	\hline
	22 & 50 & \dittotikz & 20.8 & \dittotikz & 145.6\\
	\hline
\end{tabular}
\label{tbl:Q_w_results}
\caption{Measured $Q$ and $\omega_0$ for various configurations}
\end{table}

\begin{figure}[h]
	\centering
	\includegraphics[width=\textwidth]{images/Q_w_plot}
	\label{fig:Q_w_deviation}
	\caption{Relative deviations of $Q$ and $\omega_0$}
\end{figure}

\section{Ideal drive}
This section includes simulations of the ideal drive for a 300 and 320 V drive by Chiara Decaroli.

\begin{figure}[h]
	\centering
	\includegraphics[width=\textwidth]{images/300Vcalcium}
	\label{fig:ideal_drive_300}
	\label{Simulation of the ideal drive for 300V}
\end{figure}
\begin{figure}[h]
	\centering
	\includegraphics[width=\textwidth]{images/320Vcalcium}
	\label{fig:ideal_drive_320}
	\label{Simulation of the ideal drive for 320V}
\end{figure}