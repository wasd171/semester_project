\chapter{Theory}

\section{Helical resonator models}
In order to create a helical resonator satisfying experimental conditions and limitations we inevitably come to a need for a theoretical model that would be able to predict essential characteristics of a resulting unit. The following sections aim to provide an overview and comparison of those.

\subsection{Macalpine \& Schildknecht}
\begin{figure}[h]
	\includegraphics[width=\textwidth]{images/macalpine_chart}
	\caption{Design chart for quarter-wave helical resonators.}
	\label{fig:macalpine_chart}
\end{figure}

A well-known approach \cite{Macalpine2000} for describing helical resonators was introduced in the same year as Richard Feynman's idea \cite{Feynman1960} to use quantum systems for computations. It was motivated by the possibility to reduce volume compared to TEM-mode coaxial-line resonators. While skipping a detailed theoretical analysis it nevertheless provides a basis for constructing a resonator: such as regions of usefulness, design considerations and a set of parameters' dependencies maximizing $Q$.

While describing essential properties of an unloaded helical quarter-wave resonator this paper \cite{Macalpine2000} also predicts shift of resonant frequency if an external load is connected. In order to define new frequency one can make use of telegraph equations \cite{Rohde2009} by effectively treating the ion trap as a capacitor.

\subsection{Siverns et al.}
Unfortunately modeling ion trap as a pure capacitive load is not always accurate. Introducing resistive losses imposes additional shift of resonant frequency which pushes deviation from self-resonant one even further. It is possible to tune strength of inductive coupling between antenna and main coil to compensate this shift while losing in efficiency.

These limitations of Macalpine's \& Schildknecht's \cite{Macalpine2000} model were overcome in a newer paper \cite{Siverns2012} which takes development of an amplifier specifically for the needs of quantum computing one step further. By taking a look at the joint resonator + ion trap system as a whole it aims to predict effective $Q$ and frequency. This tooling ensures that it's possible to find optimal parameters for given experimental constraints.

\section{Comparison}
Macalpine's and Schildknecht's \cite{Macalpine2000} model gives insides for designing a helical quarter-wave resonator with a given self-resonant frequency. However major shifts from it can be expected when connecting ion trap. Siverns' et al. approach \cite{Siverns2012} investigates connections between various parameters in the total circuit. As a result one could create a resonator which implements a transfer function closer to a desired one. Considering these benefits a newer model \cite{Siverns2012} was selected.