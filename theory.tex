\chapter{Theory}

\section{Helical resonator models}
In order to create a helical resonator satisfying experimental conditions and limitations we inevitably come to a need for a theoretical model that would be able to predict essential characteristics of a resulting unit. The following sections aim to provide an overview and comparison of those.

\subsection{Macalpine}
A well-known approach \cite{Macalpine2000} for describing helical resonators was introduced in the same year 1959 as Richard Feynman's idea \cite{Feynman1960} to use quantum systems for computations. It grew from the possibility to reduce volume compared to TEM-mode coaxial-line resonators.

\subsection{Hensinger}
This model \cite{Siverns2012} takes one step further to designing an amplifier specifically for the needs of quantum computing. Authors propose to take a look at the joint resonator--ion trap system as a whole. It allows to calculate proper resonant frequency, accounting for resistive losses.


\section{Comparison}

Macalpine uses a generic model for coaxial / helical resonator, later improving it by using telegrapher's equations to estimate resonant frequency shift. Hensinger takes a less generic approach, taking some resonator-only parameters from Macalpine but investigating helical resonator + ion trap system