\chapter{Mathematica code for Siverns' model}
Following calculations are heavily based on a script generously provided by James David Siverns. All variables and references correspond to \cite{Siverns2012}.
\begin{mmaCell}[moredefined={MHz, pF, mm, m, H}]{Input}
(* Units and constants *)
MHz = \mmaSup{10}{6};
pF = \mmaSup{10}{-12};
\(\Omega\) = 1;
mm = \mmaSup{10}{-3};
m = 1;
H = 1;
\(\mu\)0 = 4\(\pi\)*\mmaSup{10}{-7};
\end{mmaCell}

\begin{mmaCell}[moredefined={Cw, pF, Rt},morepattern={Ct_, Ct}]{Input}
(* Trap and wire values *)
Cw = 0.00001 pF; (* Wire to trap capacitance *)
Rt = 0.1 \(\Omega\); (* Trap resistance *)
C\(\Sigma\)[Ct_]:= Cw + Ct; (* Sum of above *)
\end{mmaCell}

\begin{mmaCell}[moredefined={calculateQ, pF, mm, m, H, Cw, MHz, Rt},morepattern={dMillimeters_, \#, dMillimeters, Ct_, Ct},morelocal={d0,
d, De, eN, b, Cc, KLc, KCs, Cs, LC, lc, r, Ns, ls, a, Rs, Rc, Rj, XLc, XCc, Xct, Xcw, XCs, Zcoil, ZE, Ztot, RealZ, Q, maxSize, log, Capacitance}]{Input}
calculateQ[dMillimeters_, \mmaPat{\(\gamma\)_}]:=Module[\{
  (* Arguments naming as in Siverns paper *)
  d0,\mmaLoc{\(\tau\)},d,De,\mmaLoc{\(\alpha\)},\mmaLoc{\(\rho\)},eN,b,Cc,KLc,KCs,Cs,LC,\mmaLoc{\(\omega\)0},\mmaLoc{\(\delta\)},lc,r,Ns,ls,a,
  Rs,Rc,Rj,XLc,XCc,Xct,Xcw,XCs,Zcoil,ZE,Ztot,RealZ,Q,
  maxSize, log, \mmaLoc{\(\omega\)Res}, Capacitance
\},
  Capacitance = 20pF;
  (* Switch log function for plotting *)
  log = Print;
  log = (#)&;
  
  (* Resonator parameters *)
  (* Coil wire diameter, we take it from Macalpine *)
  d0 = 1.95mm;
  \mmaLoc{\(\tau\)} = 2*d0; (* Winding pitch *)
  d = dMillimeters*mm; (* Diameter of coil *)
  De = d/\mmaPat{\(\gamma\)}; (* Diameter of a shield *)
  \mmaLoc{\(\alpha\)} = d/De;
  (* Resistivity of resonator material *)
  \mmaLoc{\(\rho\)} = 1.7*\mmaSup{10}{-8}*\mmaDef{\(\Omega\)}*m;  
  (* Given by the 4K chamber design *)
  maxSize = 36mm;
  b = 56mm - De/2;
  (* Handling case of a too large resonator *)
  If[d>maxSize || De>maxSize || b\(\pmb{\leq}\)0, Return@0];
  
  log["b = ", b/mm, "mm"];
  log["d = ", d/mm, "mm"];
  log["D = ", De/mm, "mm"];
  
  eN = b/\mmaLoc{\(\tau\)};(* Number of turns in the coil *)
  log["N = ", eN];
  (* Coil self capacitance - equation 25 *)
  
  Cc = ((11.26\mmaFrac{b}{d})+8+(\mmaFrac{27}{Sqrt[\mmaFrac{b}{d}]}))d pF;
  
  
  KLc = 39.37 \mmaFrac{0.025 \mmaSup{(d)}{2} (1-\mmaSup{\mmaLoc{\(\alpha\)}}{2})}{\mmaSup{(\mmaLoc{\(\tau\)})}{2}}\mmaSup{10}{-6}\mmaFrac{H}{m};
  
  KCs = 39.37 \mmaFrac{0.75}{Log[10,\mmaFrac{1}{\mmaLoc{\(\alpha\)}}]} \mmaFrac{pF}{m};
  
  (* Shield-coil capacitance - equation 26 *)
  Cs = b KCs;
  (* Inductance of coil inside a shield - equation 27 *)
  LC = b KLc;
  (* Resonant frequency - equation 21 *)
  \mmaLoc{\(\omega\)Res}[Ct_]:= \mmaFrac{1}{Sqrt[(Cs+Ct+Cw+Cc)LC]};
  
  log["\(\omega\) = ", \mmaLoc{\(\omega\)Res}[Capacitance]/(2\mmaDef{\(\pi\)} MHz), "MHz"];
  
  \mmaLoc{\(\omega\)0}[Ct_]:= 2\mmaDef{\(\pi\)}*40 MHz; (* This is a target frequency *)
  (* Allowing 5% accuracy for the frequency *)
  If[
  	\mmaFrac{Abs[\mmaLoc{\(\omega\)0}[Capacitance] - \mmaLoc{\(\omega\)Res}[Capacitance]]}{\mmaLoc{\(\omega\)0}[Capacitance]} > 0.05, 
  	Return@0
  ];
  
  \mmaLoc{\(\delta\)}[Ct_]:= Sqrt[\mmaFrac{2 \mmaLoc{\(\rho\)}}{(\mmaLoc{\(\omega\)0}[Ct] \mmaDef{\(\mu\)0} )}]; (* Skin depth *)
  
  (* Unwond length of the coil *)
  lc = 2\mmaDef{\(\pi\)} Sqrt[\mmaSup{(\mmaFrac{d}{2})}{2}+\mmaSup{(\mmaFrac{\mmaLoc{\(\tau\)}}{2\mmaDef{\(\pi\)}})}{2}]\mmaFrac{b}{\mmaLoc{\(\tau\)}};
  r = \mmaFrac{d}{2}(\mmaFrac{1}{\mmaLoc{\(\alpha\)}}-1);
  
  (* Number of "turns" in the currents path in the shield 
  - equation 31 *)
  
  Ns = \mmaFrac{b lc}{4\mmaDef{\(\pi\)} \mmaSup{r}{2}};
  
  (* Distance of current path in the shield - equation 32 *)
  ls = Ns Sqrt[\mmaSup{\mmaDef{\(\pi\)}}{2}\mmaSup{(\mmaFrac{d}{\mmaLoc{\(\alpha\)}})}{2}+\mmaSup{(\mmaFrac{b}{Ns})}{2}];
  a[Ct_]:= \mmaFrac{Ct}{Cs + Cw};
  
  Rs[Ct_]:= \mmaFrac{ \mmaLoc{\(\rho\)} ls}{b \mmaLoc{\(\delta\)}[Ct]};
  
  Rc[Ct_]:= \mmaFrac{\mmaLoc{\(\rho\)} lc}{d0 \mmaDef{\(\pi\)} \mmaLoc{\(\delta\)}[Ct]};
  
  (* Resistance of solder joint as a function of frequency - 
  the 0.003 is the DC resistance of a typical solder joint 
  between shield and coil, however this can vary 
  and is best to measure *)
  
  Rj[Ct_]:= 0.003 Sqrt[\mmaFrac{\mmaLoc{\(\omega\)0}[Ct] }{2\mmaDef{\(\pi\)} \mmaSup{10}{5}}] \mmaDef{\(\Omega\)};
  
  (* Q calculations *)
  XLc[Ct_]:= \mmaLoc{\(\omega\)0}[Ct ]LC;
  XCc[Ct_]:= \mmaFrac{1}{\mmaLoc{\(\omega\)0}[Ct] Cc};
  Xct[Ct_]:= \mmaFrac{1}{\mmaLoc{\(\omega\)0}[Ct] Ct};
  Xcw[Ct_]:= \mmaFrac{1}{\mmaLoc{\(\omega\)0}[Ct] Cw};
  XCs[Ct_]:= \mmaFrac{1}{\mmaLoc{\(\omega\)0}[Ct] Cs};
  
  Zcoil[Ct_]:= \mmaSup{(\mmaFrac{1}{(\mmaDef{i} XLc[Ct]+Rc[Ct])}+\mmaFrac{1}{\mmaFrac{1}{\mmaDef{i}} XCc[Ct]})}{-1};
  
  ZE[Ct_]:= \mmaSup{(\mmaFrac{1}{(\mmaFrac{1}{\mmaDef{i}} Xct[Ct]+Rt[Ct])}+\mmaFrac{1}{\mmaFrac{1}{\mmaDef{i}} Xcw[Ct]}+\mmaFrac{1}{\mmaFrac{1}{\mmaDef{i}}XCs[Ct]})}{-1};
  
  Ztot[Ct_]:= Zcoil[Ct] + ZE[Ct] + Rs[Ct] + Rj[Ct];
  
  RealZ[Ct_]:= \mmaFrac{Rc[Ct] \mmaSup{XCc[Ct]}{2}}{\mmaSup{Rc[Ct]}{2}+\mmaSup{(XCc[Ct]-XLc[Ct])}{2}} +
  
  \mmaFrac{Rt \mmaSup{XCs[Ct]}{2} \mmaSup{Xcw[Ct]}{2}}{\mmaSup{Rt}{2}\mmaSup{(XCs[Ct]+Xcw[Ct])}{2}+\mmaSup{(XCs[Ct](Xct[Ct]+Xcw[Ct])+Xct[Ct]Xcw[Ct])}{2}}
  
  + Rs[Ct] + Rj[Ct];
  
  Q[Ct_]:= \mmaFrac{LC \mmaLoc{\(\omega\)0}[Ct]}{RealZ[Ct]};
  
  Return@Q[Capacitance];
]
\end{mmaCell}

\begin{mmaCell}[moredefined={contourData, calculateQ},morefunctionlocal={d},morepattern={\#}]{Input}
SetDirectory[NotebookDirectory[]];
contourData = Table[
  \{\mmaFnc{\(\gamma\)}, d, calculateQ[d,\mmaFnc{\(\gamma\)}]\},
  \{d, 15, 25, 0.01\},
  \{\mmaFnc{\(\gamma\)}, 0.4, 0.7, 0.01\}
] // Flatten[#,1]&;
  
ListContourPlot[
  contourData,
  PlotLegends \(\to\) Automatic,
  FrameLabel \(\to\) \{"d/D", "d, mm"\}
]
\end{mmaCell}

\begin{mmaCell}[moredefined={calculateQ}]{Input}
(* Final parameters *)
calculateQ[19, 0.55]
\end{mmaCell}

\begin{mmaCell}{Print}
b = 38.7273mm
\end{mmaCell}
\begin{mmaCell}{Print}
d = 19mm
\end{mmaCell}
\begin{mmaCell}{Print}
D = 34.5455mm
\end{mmaCell}
\begin{mmaCell}{Print}
N = 9.93007
\end{mmaCell}
\begin{mmaCell}{Print}
\(\omega\) = 39.7919MHz
\end{mmaCell}
\begin{mmaCell}{Output}
362.188
\end{mmaCell}