\chapter{Conclusion}

As a result of this project we were able to implement a delicate helical resonator with reasonable $Q$ and $\omega_0$ values. There is definitely room for improvement:

\begin{itemize}
	\item Angular tuning didn't prove to work, which is also good since it shows that structure is fairly symmetrical to rotations.
	\item Distance tuning almost didn't work, for an optimal tuning the distance between the antenna mount was kept close to 0. Possible solution would be either to add another turn to the antenna itself or extend the length of the antenna mount to gain more flexibility.
	\item Helical resonator has spent a fair share of time being in contact with air. Cleaning the internal surface of the shield with sanding or acid should remove oxidized layer of copper and improve conductivity.
	\item SMA adapters are good for testing and probing the device, but add additional capacitance ––– for a usage with a real trap it would be recommended to solder the wires directly.
	\item Coaxial cables also add capacitance, better results might be achieved with using regular wires.
	\item Theoretical calculations used conductivity at cryogenic temperatures, moving the helical resonator into the cryostat should provide results closer to those expected from the model.
\end{itemize}