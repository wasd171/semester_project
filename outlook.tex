\chapter{Outlook}

In this chapter we would respond to the issues lifted by the chapter \ref{chapter:results} and share tips for the future similar projects.

\begin{itemize}
	\item Ideally initial $\omega_0$ for the algorithm defined in the appendix \ref{chapter:macalpine_code} should be picked as a middle value in the intervals discussed in the section \ref{section:ideal_drive}. Afterwards in the appendix \ref{chapter:siverns_code} instead of a percentage frequency margin we check whether the resulting frequency belongs to the corresponding interval.
	\item Cleaning the outer oxide layer by sanding or acid and placing the resonator into the cryostat should lower the resistivity of the skin layer thus improving $Q$ \cite{Leupold2015, Fadel2013}
	\item Model should account better for the SMA connectors and wires. Just like with the trap capacitance and the resistance of the solder joint, approximate values should be used for the calculations
	\item Removing additional capacitance and resistance by soldering the non-coaxial cable directly to the shield might yield better results
	\item In order to improve the distance-based tuning the length of the antenna mount can be increased and/or antenna should contain more turns to strengthen the antenna to coil coupling
	\item Lack of the angular tuning means that the resonator was produced with high precision since it shows that the system has a circular symmetry. However this can be also verified without the angular scale, which allows us to drop it in the new designs
\end{itemize}